% Options for packages loaded elsewhere
\PassOptionsToPackage{unicode}{hyperref}
\PassOptionsToPackage{hyphens}{url}
%
\documentclass[
]{article}
\usepackage{amsmath,amssymb}
\usepackage{lmodern}
\usepackage{ifxetex,ifluatex}
\ifnum 0\ifxetex 1\fi\ifluatex 1\fi=0 % if pdftex
  \usepackage[T1]{fontenc}
  \usepackage[utf8]{inputenc}
  \usepackage{textcomp} % provide euro and other symbols
\else % if luatex or xetex
  \usepackage{unicode-math}
  \defaultfontfeatures{Scale=MatchLowercase}
  \defaultfontfeatures[\rmfamily]{Ligatures=TeX,Scale=1}
\fi
% Use upquote if available, for straight quotes in verbatim environments
\IfFileExists{upquote.sty}{\usepackage{upquote}}{}
\IfFileExists{microtype.sty}{% use microtype if available
  \usepackage[]{microtype}
  \UseMicrotypeSet[protrusion]{basicmath} % disable protrusion for tt fonts
}{}
\makeatletter
\@ifundefined{KOMAClassName}{% if non-KOMA class
  \IfFileExists{parskip.sty}{%
    \usepackage{parskip}
  }{% else
    \setlength{\parindent}{0pt}
    \setlength{\parskip}{6pt plus 2pt minus 1pt}}
}{% if KOMA class
  \KOMAoptions{parskip=half}}
\makeatother
\usepackage{xcolor}
\IfFileExists{xurl.sty}{\usepackage{xurl}}{} % add URL line breaks if available
\IfFileExists{bookmark.sty}{\usepackage{bookmark}}{\usepackage{hyperref}}
\hypersetup{
  pdftitle={Mechanical Metallurgy},
  pdfauthor={Joby M. Anthony III},
  pdfsubject={ENGR 839-001},
  hidelinks,
  pdfcreator={LaTeX via pandoc}}
\urlstyle{same} % disable monospaced font for URLs
\setlength{\emergencystretch}{3em} % prevent overfull lines
\providecommand{\tightlist}{%
  \setlength{\itemsep}{0pt}\setlength{\parskip}{0pt}}
\setcounter{secnumdepth}{-\maxdimen} % remove section numbering
\usepackage{enumitem}
\setlistdepth{9}

\setlist[itemize,1]{label=$\bullet$}
\setlist[itemize,2]{label=$\bullet$}
\setlist[itemize,3]{label=$\bullet$}
\setlist[itemize,4]{label=$\bullet$}
\setlist[itemize,5]{label=$\bullet$}
\setlist[itemize,6]{label=$\bullet$}
\setlist[itemize,7]{label=$\bullet$}
\setlist[itemize,8]{label=$\bullet$}
\setlist[itemize,9]{label=$\bullet$}
\renewlist{itemize}{itemize}{9}

\setlist[enumerate,1]{label=$\arabic*.$}
\setlist[enumerate,2]{label=$\alph*.$}
\setlist[enumerate,3]{label=$\roman*.$}
\setlist[enumerate,4]{label=$\arabic*.$}
\setlist[enumerate,5]{label=$\alpha*$}
\setlist[enumerate,6]{label=$\roman*.$}
\setlist[enumerate,7]{label=$\arabic*.$}
\setlist[enumerate,8]{label=$\alph*.$}
\setlist[enumerate,9]{label=$\roman*.$}
\renewlist{enumerate}{enumerate}{9}
\ifluatex
  \usepackage{selnolig}  % disable illegal ligatures
\fi

\title{Mechanical Metallurgy}
\author{Joby M. Anthony III}
\date{210823}

\begin{document}
\maketitle

\hypertarget{engr-839-001-mechanical-metallurgy}{%
\section{ENGR 839-001: Mechanical
Metallurgy}\label{engr-839-001-mechanical-metallurgy}}

\begin{itemize}
\tightlist
\item
  \protect\hyperlink{engr-839-001-mechanical-metallurgy}{ENGR 839-001:
  Mechanical Metallurgy}

  \begin{itemize}
  \tightlist
  \item
    \protect\hyperlink{materials-structure-properties-and-performance}{Materials:
    Structure, Properties, and Performance}

    \begin{itemize}
    \tightlist
    \item
      \protect\hyperlink{introduction}{Introduction}
    \item
      \protect\hyperlink{review}{Review}
    \item
      \protect\hyperlink{metallic-crystal-structure}{Metallic Crystal
      Structure}

      \begin{itemize}
      \tightlist
      \item
        \protect\hyperlink{energy-and-packing}{\{Energy and Packing\}}
      \item
        \protect\hyperlink{crystalline-periodic-structure}{\{Crystalline
        (Periodic) Structure\}}
      \end{itemize}
    \item
      \protect\hyperlink{density-comparison-of-materials}{Density
      Comparison of Materials}

      \begin{itemize}
      \tightlist
      \item
        \protect\hyperlink{polycrystalline-materials}{\{Polycrystalline
        Materials\}}
      \item
        \protect\hyperlink{anistropy}{\{Anistropy:\}}
      \item
        \protect\hyperlink{istropy}{\{Istropy\}}
      \item
        \protect\hyperlink{miller-indices}{\{Miller Indices\}}

        \begin{itemize}
        \tightlist
        \item
          \protect\hyperlink{point-coordinates-algorithm}{\{Point
          coordinates Algorithm\}}
        \item
          \protect\hyperlink{crystallographic-directions-algorithm}{\{Crystallographic
          Directions Algorithm\}}
        \item
          \protect\hyperlink{crystallographic-planes-algorithm}{\{Crystallographic
          Planes Algorithm\}}
        \end{itemize}
      \item
        \protect\hyperlink{crystal-structure-and-deformation}{\{Crystal
        Structure and Deformation\}}
      \item
        \protect\hyperlink{slip-systems-fcc}{\{Slip Systems (fcc)\}}
      \item
        \protect\hyperlink{polycrystalline-slip}{\{Polycrystalline
        Slip\}}
      \end{itemize}
    \item
      \protect\hyperlink{summary}{Summary}
    \end{itemize}
  \item
    \protect\hyperlink{elasticity}{Elasticity}

    \begin{itemize}
    \tightlist
    \item
      \protect\hyperlink{introduction-1}{Introduction}
    \item
      \protect\hyperlink{elastic}{Elastic}

      \begin{itemize}
      \tightlist
      \item
        \protect\hyperlink{stress}{\{Stress\}}
      \item
        \protect\hyperlink{strain}{\{Strain\}}
      \item
        \protect\hyperlink{engineering-vs-true-stress}{\{Engineering vs
        True Stress\}}
      \item
        \protect\hyperlink{notation}{\{Notation!\}}
      \item
        \protect\hyperlink{strain-energy}{\{Strain Energy\}}
      \item
        \protect\hyperlink{sheartorsion}{\{Shear/Torsion\}}
      \item
        \protect\hyperlink{poissons-ratio}{\{Poisson's Ratio\}}
      \item
        \protect\hyperlink{summary-1}{\{Summary\}}
      \end{itemize}
    \item
      \protect\hyperlink{elasticity-polycrystalline-materials-and-bulk-metals}{Elasticity:
      Polycrystalline Materials and Bulk Metals}

      \begin{itemize}
      \tightlist
      \item
        \protect\hyperlink{introduction-2}{\{Introduction\}}
      \item
        \protect\hyperlink{stress-tensor-revisited}{\{Stress Tensor
        Revisited\}}
      \item
        \protect\hyperlink{hookes-law-revisited}{\{Hooke's Law
        Revisited\}}
      \item
        \protect\hyperlink{simplifications}{\{Simplifications\}}
      \item
        \protect\hyperlink{mohrs-circle-revisited}{\{Mohr's Circle
        Revisited\}}
      \end{itemize}
    \item
      \protect\hyperlink{pure-shear}{Pure Shear}
    \item
      \protect\hyperlink{elasticity-atomic-bonds}{Elasticity: Atomic
      Bonds}

      \begin{itemize}
      \tightlist
      \item
        \protect\hyperlink{introduction-3}{\{Introduction\}}
      \item
        \protect\hyperlink{atomic-bonding}{\{Atomic Bonding\}}
      \item
        \protect\hyperlink{bond-interaction-and-force}{\{Bond
        Interaction and Force\}}
      \item
        \protect\hyperlink{stress-and-strain-in-bonds}{\{Stress and
        Strain in Bonds\}}
      \item
        \protect\hyperlink{summary-2}{\{Summary\}}
      \end{itemize}
    \end{itemize}
  \item
    \protect\hyperlink{exam-review}{*\{Exam Review\}}
  \item
    \protect\hyperlink{plasticity}{Plasticity}

    \begin{itemize}
    \tightlist
    \item
      \protect\hyperlink{introduction-4}{Introduction}

      \begin{itemize}
      \tightlist
      \item
        \protect\hyperlink{engineering-and-true-stress-and-strain}{\{Engineering
        and true stress and strain\}}
      \item
        \protect\hyperlink{work-hardening-basics}{\{Work-Hardening
        Basics\}}
      \item
        \protect\hyperlink{summary-3}{\{Summary\}}
      \end{itemize}
    \item
      \protect\hyperlink{tensile-curve-parameters-necking-and-strain-rate}{Tensile
      Curve Parameters, Necking, and Strain Rate}

      \begin{itemize}
      \tightlist
      \item
        \protect\hyperlink{introduction-5}{\{Introduction\}}
      \item
        \protect\hyperlink{necking}{\{Necking\}}
      \item
        \protect\hyperlink{stress-strain-and-necking}{\{Stress-strain
        and Necking\}}

        \begin{itemize}
        \tightlist
        \item
          \protect\hyperlink{bridgman-correction}{\{Bridgman
          Correction\}}
        \item
          \protect\hyperlink{state-of-stress-in-deformation}{\{State of
          Stress in Deformation\}}
        \end{itemize}
      \item
        \protect\hyperlink{strain-rate}{\{Strain Rate\}}
      \item
        \protect\hyperlink{summary-4}{\{Summary\}}
      \end{itemize}
    \item
      \protect\hyperlink{compression-and-hardness}{Compression and
      Hardness}

      \begin{itemize}
      \tightlist
      \item
        \protect\hyperlink{introduction-6}{\{Introduction\}}
      \item
        \protect\hyperlink{practical-considerations}{\{Practical
        Considerations\}}
      \item
        \protect\hyperlink{compression-curve}{\{Compression Curve\}}
      \item
        \protect\hyperlink{compression-failure}{\{Compression Failure\}}
      \item
        \protect\hyperlink{bauschinger-effect}{\{Bauschinger Effect\}}
      \item
        \protect\hyperlink{hardness-testing}{\{Hardness Testing\}}
      \end{itemize}
    \end{itemize}
  \end{itemize}
\end{itemize}

\hypertarget{materials-structure-properties-and-performance}{%
\subsection{Materials: Structure, Properties, and
Performance}\label{materials-structure-properties-and-performance}}

\hypertarget{introduction}{%
\subsubsection{Introduction}\label{introduction}}

\textbf{Objectives} The course content will enable students to set and
meet expectations. All of this material, in its content and instruction,
are new: so, leave room for error. Be inspired to tackle whatever comes
our way.

{Creationeering diagram.}

{Cross-shaped Liberty University Creationeer.}

\textbf{Definitions} This class will make more familiar the
``chemistry-process-structure'' portion of the
\textit{chemistry-process-structure-property-performance}" relationship.
LU seeks to apply these engineering disciplines into entrepreneurship.
\textbf{Mechanical}: force response of materials, which includes the
underlying principles/effects of microstructure. \textbf{Metallurgy}:
microstructural transformations dictated by composition/processing that
is the foundation. \textbf{Mechanical Metallurgy}, as a course, studies
not \textit{how} materials behave, by \textit{why} they behave a certain
way.

\textbf{Syllabus:} Homeworks due only before the relevant exams, and
graded for the good ol' college try. If you can do the homeworks, then
you can do the tests. \textbf{Final Exam is cummulative.} Critical
Review: ``Reverse engineer'' an article and explain it with additional
comments or inferences. The accompanying presentation should explain the
takeaway from the application articles/comments.

\hypertarget{review}{%
\subsubsection{Review}\label{review}}

\textbf{\textbf{CPSPP}:} Relationship for how some chemical composition
(e.g: FeC) is processed to produce some structure, which has some
property that allows some performance.

{Iron-Carbon phase diagram.}

{Microstructures in FeC phase diagram. Phase-Diagrams pack a lot of
information for CPS of CPSPP sequence. Lines denote what structure is
found after some process with some chemical composition.
\textit{Lever Rule}: ratio of phases proportional to weight percent
distance to other phases.}

{Crystallographic examples.}

\textbf{Crystallographic Types:} simple cubic, face-center cubic (fcc),
body-center cubic (bcc), hexagonal close-packed (hcp). The atomic
packing factor, apf--the ratio lattice that is filled--increases down
this list. fcc and hcp are similar, but hcp is more brittle.

{Copper phase diagram.}

{The angle at which one stuffs at burger into his face.}

\textbf{\textbf{Berger's vector, $\vec{b}$}:} lattice displacement by a
dislocation; vector of dislocation; location, magnitude, and direction
of influence on lattice. Can weaken metals when many dislocations exist.

{Pictographic way to visual CPSPP relationship according to processing
techniques.}

\textbf{CPSPP Example:} includes low-carbon steel; high-carbon steel;
and, cast iron. All same chemistry with different composition ratios and
examples of different processing techniques applied/able.

{Common steels in automotive applications.}

\textbf{Monolithic} Previously explained homogeneous or isotropic, but
depends on length scale. Here, means the same crystal structure
throughout: e.g.~pearlite.

{Phase diagram for FeC with microstructure.}

\textbf{Materials Design} Exposed to variety of conditions, but can be
designed to suit. More complex designs may have better performance.
Composites have ``rules-of-mixtures'' to balance performance.
Hierarchical materials vary structure/composition to enhance
functional/structural properties at different length scales.

\textbf{Our Focus:} not on manufacturing techniques: e.g.~ENGR 835. Will
consider how manufacturing variables affect structure. We will move up
length up from atomistics.

\hypertarget{metallic-crystal-structure}{%
\subsubsection{Metallic Crystal
Structure}\label{metallic-crystal-structure}}

\hypertarget{energy-and-packing}{%
\paragraph{\{Energy and Packing\}}\label{energy-and-packing}}

\leavevmode\hypertarget{fig:bonding_energy_curves}{}%
{Non-dense, random packing.}

{Dense, ordered packing.}

Ordered structures tend to be nearer the minimum bonding energy and are
more stable.

\textbf{Atomic Packing} Dense crystal structures for metals. Reasons for
dense packing: \textit{1)}bonds between metal atoms are non-directional;
\textit{2)} nearest neighbor distances tend to be small in order to
lower bond energy; \textit{3)} high degree of shielding (of ion cores)
provided by free electron cloud; and, \textit{4)} crystal structures for
metals simpler than structures for ceramics and polymers.

\hypertarget{crystalline-periodic-structure}{%
\paragraph{\{Crystalline (Periodic)
Structure\}}\label{crystalline-periodic-structure}}

{Crystalline (Periodic) Structure.}

{Simple Cubic (sc) Crystal Structure: centers of atoms at corners and
close packed along edges.}

{Body-Centered Cubic (bcc) Structure: atoms located at 8 cube corners
with a single atom at cube center.}

{Face-Centered Cubic (fcc) Structure: atoms located at 8 cube corners
with half atoms at center of 6 cube faces.}

{Hexagonal Close-Packed (hcp) Structure: only B or C positions can be
filled in a single layer.}

{Interces in each layer can be filled only according to some pattern:
``A'' or ``B''.}

\hypertarget{density-comparison-of-materials}{%
\subsubsection{Density Comparison of
Materials}\label{density-comparison-of-materials}}

{In general, \(\rho_{metals} > \rho_{ceramics} > \rho_{polymers}\)}

\hypertarget{polycrystalline-materials}{%
\paragraph{\{Polycrystalline
Materials\}}\label{polycrystalline-materials}}

{Higher cooling rates at edges make for smaller grains; whereas, slower
cooling rate at center make for larger, directionally aligned grains.}

\textit{Most} engineering materials are composed of many, small single
crystals. Each ``grain'' is a single crystal. Grain sizes typically
range from 1 nm to 2 cm. Smaller grains usually mean higher strength and
lower ductility.

{Same crystal type that repeat in different orientations.}

\hypertarget{anistropy}{%
\paragraph{\{Anistropy:\}}\label{anistropy}}

{Unit cell of BCC (\(\alpha\)) iron.}

A property value that depends on crystallographic direction of
measurement. Properties depend on direction loading, because of linear
density along that direction. Best observed in single crystals. The
higher linear density means higher stiffness, usually.

\hypertarget{istropy}{%
\paragraph{\{Istropy\}}\label{istropy}}

{Randomly oriented versus textured grains.}

\textbf{Polycrystals:} properties may (not) vary with direction. If
grains randomly oriented, properties are isotropic: independent of
loading direction. If grains, ``textured''--crystallographic
orientation--properties are anistropic: dependent on loading direction.

\hypertarget{miller-indices}{%
\paragraph{\{Miller Indices\}}\label{miller-indices}}

A method of assigning coordinate values to crystallographic lattice
sites. Used to identify specific points, directions, and planes or
families of these. Used to identify crystallographic information. Three
ways to identify:

\hypertarget{point-coordinates-algorithm}{%
\subparagraph{\{Point coordinates
Algorithm\}}\label{point-coordinates-algorithm}}

\begin{verbatim}
**NOTE** A lattice position with a unit cell and determined as fractional multiples of unit cell edge lengths.
\end{verbatim}

\leavevmode\hypertarget{fig:point_coordinate_example}{}%
{Determined as fractional multiples of unit cell edge lengths.}

{}

\begin{enumerate}
\def\labelenumi{\arabic{enumi}.}
\tightlist
\item
  Lattice position is a, b, c.
\item
  Divide by unit cell edge lengths and remove commas:
  \(\frac{a}{a}\frac{b}{b}\frac{c}{c} = 111\)
\end{enumerate}

\hypertarget{crystallographic-directions-algorithm}{%
\subparagraph{\{Crystallographic Directions
Algorithm\}}\label{crystallographic-directions-algorithm}}

{}

\begin{verbatim}
**NOTE** Remember from calculus the $tip - tail$ method to find vector direction and length.
\end{verbatim}

\begin{enumerate}
\def\labelenumi{\arabic{enumi}.}
\item
  Determine coordinates of vector head and tail:
  \((x_{2}, y_{2}, z_{2})\) and \((x_{1}, y_{1}, z_{1})\), respectively.
\item
  Subtract tail coordinates from head coordinates.
\item
  Normalize this subtraction by lattice parameters of unit cell edge
  length:
  \(\frac{x_{2} - x_{1}}{a}\frac{y_{2} - y_{1}}{b}\frac{z_{2} - z_{1}}{c}\).
\item
  Multiply to smallest integer values.
\item
  Enclose in square brackets with no commas: \([uvw]\).

  \textbf{NOTE} Negative indices represented with overbars:
  \(-4, 1, 2 \implies [\bar{4}12]\) A \emph{family of directions} are
  crystallographically equivalent (same atomic spacing) and indicated by
  angle brackets, \(<>\).
\end{enumerate}

{Common crystallographic directions.}

\hypertarget{crystallographic-planes-algorithm}{%
\subparagraph{\{Crystallographic Planes
Algorithm\}}\label{crystallographic-planes-algorithm}}

\begin{enumerate}
\def\labelenumi{\arabic{enumi}.}
\tightlist
\item
  If plane passes through origin, establish a new origin in another unit
  cell.
\item
  Read off values of intercepts of plane (designated \(A, B, C\)) with
  \(x, y, z\) axes in terms of \(a, b, c\).
\item
  Take reciprocals of intercepts.
\item
  Normalize reciprocals by multiplying lattice parameters \(a, b, c\).
\item
  Reduce to smallest integer values.
\item
  Enclose resulting indices in parentheses without commas:
  i.e.~\((hkl)\).
\end{enumerate}

{Resulting Miller Indices: (110).}

{Resulting Miller Indices: (200).}

\begin{verbatim}
**NOTE** A family of planes cannot be reduced any simpler than LCM. Family may be parallel to other families and planar densities may be equivalent, but linear density will vary.
\end{verbatim}

{Pay attention to origin!}

{Resulting Miller Indices: (634).}

\begin{verbatim}
**NOTE** A *family of planes* are crystallographically equivalent (same APF) and are indicated by indices in braces, $\{\}$.
\end{verbatim}

\begin{verbatim}
**EXAMPLE** For HCP, determine intercepts with $a_{1}, a_{2}$ and $z$ axes, then determine the Miller-Bravais indices $h, k, i, l$.
\end{verbatim}

{Resulting Miller Indices: \((10\bar{1}1)\).}

{Projections are parallel to edge.}

{\textit{Planar Density (PD) of Atoms},
\(PD = \frac{\#~of~atoms~centered~on~plane}{area~of~plane}\).}

\hypertarget{crystal-structure-and-deformation}{%
\paragraph{\{Crystal Structure and
Deformation\}}\label{crystal-structure-and-deformation}}

Planar density determine slip planes: the more dense, the easier to
move. Unit cell represents single crystal, but the behavior of that
structure is not equivalent in all directions. Permanently deforming
materials requires that atoms must shift over one another:
\textit{slip planes}. Closely packed (high apf) do not have to move as
much to get by one another. Not all slip directions\ldots{} fcc
materials are generally ductile, because of few slip systems and one
slip plane. Not many preferred directions to slip and few preferred:
e.g.~48 spoons to cut a steak does not equal cutting with a knife. More
of a bad thing does not outweigh few good things. Coordination number is
the number of nearest neighboring atoms.

\hypertarget{slip-systems-fcc}{%
\paragraph{\{Slip Systems (fcc)\}}\label{slip-systems-fcc}}

{}

\textbf{Slip system is \({111}<110>\)} Dislocation motion on \({111}\)
planes. Dislocation motion in \(<110>\) directions. 12 indepedent slip
systems for fcc.

{More force over more distance = more work to move bcc than fcc, which
requires less distance because closely packed (dense).}

\hypertarget{polycrystalline-slip}{%
\paragraph{\{Polycrystalline Slip\}}\label{polycrystalline-slip}}

Many grains, often with random crystallographic directions. Orientation
of slip planes and slip directions, \((\phi, \lambda)\) vary from grain
to grain. On application of stress, slip in each grain on most favorable
slip system. With largest \(T_{R}\). When \(T_{R} > T_{crss}\).

{}

{Edge of indent shows texture from slip planes from moving atoms out of
the way of impact. Accumulates with amount of deformation applied.}

\hypertarget{summary}{%
\subsubsection{Summary}\label{summary}}

Important to have common understanding of vocabulary to describe crystal
structure. We focus on \emph{bcc, fcc,} and \emph{hcp} structures.
Miller indices give exact orientation or active atoms/planes to
determine the associated properties. The packing factor is import to
density and slip systems, but it is not the only critical property.

\hypertarget{elasticity}{%
\subsection{Elasticity}\label{elasticity}}

\hypertarget{introduction-1}{%
\subsubsection{Introduction}\label{introduction-1}}

Stresses occur at varying scales within a material. Macro-stresses
include component design and assemblage. Micro-stresses are found within
the material and include those from defects: dislocations, alloying
elements. etcetera. Stresses at smallest scale act cumulatively to
produce the response to the largest scale. This all relates to the
\textbf{CPSPP} relationship.

\hypertarget{elastic}{%
\subsubsection{Elastic}\label{elastic}}

\begin{itemize}
\tightlist
\item
  Reverse deformation, that instantaneously recovers to its original
  dimensions after force is removed.
\item
  Analogous to a spring, the relationship was conveniently described by
  Robert Hooke: \(F = k\delta x\)
\end{itemize}

\hypertarget{stress}{%
\paragraph{\{Stress\}}\label{stress}}

\begin{itemize}
\tightlist
\item
  Stress is the result of applied force and response of material to
  balance external force.

  \begin{itemize}
  \tightlist
  \item
    Area resisting force is perpendicular to applied force line of
    action.
  \item
    Material response is determined by atomic bonds and orientation
    relative to crystal lattice. {[}NOTE{]} Polycrystalline materials
    assumed anisotropic.
  \item
    Component response dictated by design and material properties.
  \item
    Tensile stress often positive in sign and compression is negative,
    by convention.
  \end{itemize}
\end{itemize}

\hypertarget{strain}{%
\paragraph{\{Strain\}}\label{strain}}

\begin{itemize}
\tightlist
\item
  Physical result of stress.
\item
  Linear proportional to stress in Hookean material up to its elastic
  limit/yield stress. {[}NOTE{]} Proportional limit is that when the
  sress-strain deviates from linear and yield stress is at the 0.02\%
  strain offset.
\item
  Metals often Hookean in nature and assumed until otherwise stated.
\item
  True strain:
  \(\epsilon = \frac{dl}{l}, \epsilon = \int_{l_{0}}^{l_{1}}\frac{dl}{l} = ln(\frac{l_{1}}{l_{0}})\)
\item
  Engineering strain: \(\epsilon = \frac{\delta l}{l}\)
\end{itemize}

\hypertarget{engineering-vs-true-stress}{%
\paragraph{\{Engineering vs True
Stress\}}\label{engineering-vs-true-stress}}

\begin{itemize}
\tightlist
\item
  Comes from initial conditions and easily measured. {[}NOTE{]} Area
  does not change, so the initial area determines the stress throughout
  the entire deformation.
\item
  True stress and strain are more accurate and use an instantaneous
  cross-section.
\item
  In the elastic region for metals, deformation is typically small and
  engineering and true stress-strain values are comparable.
\item
  \(\sigma_{T} = \sigma(1 + \epsilon)\)
\item
  \(\epsilon_{T} = ln(1 + \epsilon)\)
\end{itemize}

{}

\hypertarget{notation}{%
\paragraph{\{Notation!\}}\label{notation}}

\begin{itemize}
\tightlist
\item
  Notation is not always the same.
\item
  Tensor notation:
\end{itemize}

{}

\hypertarget{strain-energy}{%
\paragraph{\{Strain Energy\}}\label{strain-energy}}

\begin{itemize}
\tightlist
\item
  Work done during deformation is converted to heat and internal energy.
\item
  \(W = Fd\), but \(F\) is not constant!
\item
  Elastic deformation does not typically produce much heat, so most is
  stored internally.
\item
  \(dU = dQ - dQ\)
\item
  \(U = W\) (without heat)
\item
  \(W = \frac{1}{2}\sigma_{ij}\epsilon_{ij} = \frac{1}{2}E\epsilon_{ij}^{2} = \frac{1}{2E}\sigma_{ij}^{2}\)
\end{itemize}

\hypertarget{sheartorsion}{%
\paragraph{\{Shear/Torsion\}}\label{sheartorsion}}

\begin{itemize}
\tightlist
\item
  Shear and torsional stress are handled similarly, but vary by
  configuration.
\end{itemize}

{For a cylindrical punch, the average diameter of the punch and hole can
be taken, and the area will be the circumference times the thickness of
the material.}

{For torsion of a rod, the stress and strain vary radially and axially.}

\begin{itemize}
\tightlist
\item
  \(\tau = \frac{F}{A}\)
\item
  \(\gamma = \frac{dl}{l} = tan\theta \cong \theta\)
\item
  \(G = \frac{\tau}{\gamma}\)
\end{itemize}

\textbf{Q\&A} Why does \(\gamma \cong \theta\)? Small angle assumptions
in radians mean that \(sin\theta = \theta\).

\hypertarget{poissons-ratio}{%
\paragraph{\{Poisson's Ratio\}}\label{poissons-ratio}}

\(\nu = -\frac{\epsilon_{11}}{\epsilon_{33}} = -\frac{\epsilon_{22}}{\epsilon_{33}}\)

{}

\begin{itemize}
\tightlist
\item
  Describes the consequent strains orthogonal to an applied stress.
\item
  Most metals are 0.3.
\item
  For constant volume without lateral contraction, the ratio is 0.5
  (plastic only).
\end{itemize}

\hypertarget{summary-1}{%
\paragraph{\{Summary\}}\label{summary-1}}

\begin{itemize}
\tightlist
\item
  Elasticity affects everything, because it is initial response to
  stress.
\item
  Material behavior in elastic region dictates behavior in application.
\item
  Components rarely useful in plastic region, so they will spend their
  lifetime in the elastic range.
\item
  Although elastic properties are commonly discussed at a continuum
  scale, we will find it important to atomic scale phenomena.
\end{itemize}

\hypertarget{elasticity-polycrystalline-materials-and-bulk-metals}{%
\subsubsection{Elasticity: Polycrystalline Materials and Bulk
Metals}\label{elasticity-polycrystalline-materials-and-bulk-metals}}

\hypertarget{introduction-2}{%
\paragraph{\{Introduction\}}\label{introduction-2}}

\begin{itemize}
\tightlist
\item
  Stress may be applied in any arrangement, and we can determine the
  overall conditions of strain.
\item
  This is often reduced to two dimensions, but it does not have to be
  (it is just simpler).
\item
  We can determine the state of stress as a point in any orientation
  (not just aligned with the applied stress) by using transformation
  techniques.
\item
  Certain orientations will result in only normal or shear stresses,
  which can be important for determining the resulting performance.
\end{itemize}

\hypertarget{stress-tensor-revisited}{%
\paragraph{\{Stress Tensor Revisited\}}\label{stress-tensor-revisited}}

{}

\begin{itemize}
\tightlist
\item
  Cuachy stress tensor describes the complete state of stress in three
  dimensions.
\item
  Valid for small deformations.
\end{itemize}

\hypertarget{hookes-law-revisited}{%
\paragraph{\{Hooke's Law Revisited\}}\label{hookes-law-revisited}}

\begin{itemize}
\tightlist
\item
  Linear elastic response in homogeneous metals and alloys which allows
  us to relate stress and strain in three dimensions through the Poisson
  ratio, \(\nu\)
\item
  \(\epsilon_{11} = \frac{1}{E}[\sigma_{11} - \nu(\sigma_{22} + \sigma_{33})]\)
\item
  \(\epsilon_{22} = \frac{1}{E}[\sigma_{22} - \nu(\sigma_{11} + \sigma_{33})]\)
\item
  \(\epsilon_{33} = \frac{1}{E}[\sigma_{33} - \nu(\sigma_{11} + \sigma_{22})]\)
\item
  \(\gamma_{12} = \frac{\sigma_{12}}{G}\)
\item
  \(\gamma_{13} = \frac{\sigma_{13}}{G}\)
\item
  \(\gamma_{23} = \frac{\sigma_{23}}{G}\)
\end{itemize}

\textbf{Example 1:} For the following stress tensor, what are the
resulting normal strains if \(E = 10e3 kip\) and \(\nu = 0.3\)?

\(\sigma = 5, 3, 2\)\(, [3, -1, 0\)\(, [2, 0, 4\)\$\$\(kip\)

\(\epsilon_{normal} = [\frac{1}{10e3 ksi}[5 ksi - 0.3(-1 + 4)ksi]], [\frac{1}{10e3 ksi}[1 ksi - 0.3(5 + 4)ksi]\)\(, [\frac{1}{10e3 ksi}[4 ksi - 0.3(5 + (-1))ksi]\ = 4.1e-4\)\(, [-3.7e-6\)\(, [2.8e-6]\ \)

\hypertarget{simplifications}{%
\paragraph{\{Simplifications\}}\label{simplifications}}

\begin{itemize}
\tightlist
\item
  To avoid complex tridimensional description of stress and strain,
  especially during plastic deformation, simplifications are possible.

  \begin{itemize}
  \tightlist
  \item
    Plane stress: strain and strain go to zero at free surface (normal
    and shear).
  \item
    Plane strain: one dimension is considered infinite.
  \end{itemize}
\item
  Conditions of pure shear can also be considered, such that no normal
  forces are present.
\end{itemize}

\hypertarget{mohrs-circle-revisited}{%
\paragraph{\{Mohr's Circle Revisited\}}\label{mohrs-circle-revisited}}

\begin{itemize}
\tightlist
\item
  A graphical way to represent stress transformation to alternative
  orientations.
\item
  These are set of equations that describe stress transformation, but a
  graphical solution is often useful.
\end{itemize}

{}

\begin{itemize}
\tightlist
\item
  Principal stress: maximum/minimum normal stresses (no shear)
\end{itemize}

{}

\begin{itemize}
\tightlist
\item
  \(R = \sqrt{(\frac{\sigma_{11} + \sigma_{22}}{2})^{2} + \tau_{12}^{2}}\)
\item
  \(\sigma_{11}' = \frac{\sigma_{11} + \sigma_{22}}{2} + \frac{\sigma_{11} - \sigma_{22}}{2}cos(2\theta) + \sigma_{12}sin(2\theta)\)
\item
  \(\sigma_{12}' = -\frac{\sigma_{11} - \sigma_{22}}{2}sin(2\theta) + \sigma_{12}cos(2\theta)\)
\end{itemize}

\textbf{Example 1:} What is the stress state if rotated
\(30\text{\textdegree}\) CCW?

{}

\emph{Given:
\(\sigma_{11} = -8 ksi, \sigma_{22} = 12 ksi, \sigma_{12} = -6 ksi\)}

Center point,
\(C = \sigma_{avg} = \frac{\sigma_{11} + \sigma_{22}}{2} = \frac{-8 + 12}{2} = 2 ksi\)

{}

\(R = \sqrt{10^{2} + 6^{2}} = 11.66\)

{}

Use trig to find \(\phi\) and \(\psi\):
\(\phi = tan^{-1}(\frac{6}{10}) = 30.96\text{\textdegree}, \psi = 60 - \phi = 29.04\text{\textdegree}\)

\begin{itemize}
\tightlist
\item
  \(\sigma_{11}' = 2 - 11.66cos(29.04) = -8.2 ksi\)
\item
  \(\sigma_{12}' = -11.66sin(29.04) = 5.66 ksi\)
\end{itemize}

{Slight drop increase of normal stress and slight decrease of shear
stress.}

{Basic steps to find all stresses and strains at orientations relative
to what you start with.}

\hypertarget{pure-shear}{%
\subsubsection{Pure Shear}\label{pure-shear}}

There exists a condition which has only shear stresses and no normal
stress. This condition is called \textit{pure shear}, which means that
\(\sigma_{11} = -\sigma_{22}\). This implies that Mohr's circle is
centered at the origin, because \(\sigma_{avg} = 0\). Graphically, the
maximum shear stress possible in this condition is the radius of Mohr's
circle and at \(90\text{\textdegree}\) from the horizontal. However,
recall that angles in Mohr's circle are twice real angles: e.g.~if
\(\theta_{Mohr} = 2*\theta_{real}\) and
\(\theta_{Mohr} = \text{\textdegree}90\), then
\(\theta_{real} = 45\text{\textdegree}\).

{If a component experiences compression in one axis and tension in a
perpendicular axis, then elements within the component experience pure
shear.}

Using knowledge of this condition, we can find stress and strain
information with the following relations:

\begin{align}
\epsilon_{11} &= \frac{1}{E}(\sigma_{1} - \nu\sigma_{2}) = \frac{\sigma_{1}}{E}(1 + \nu) \\
\tau &= -\sigma_{1} (on~circle,~with~sign~convention) \\
\tau &= G\gamma \\
\epsilon_{11} &= -\frac{G\gamma}{E}(1 + \nu) \\
2\epsilon_{11} &= -\gamma \\
G &= \frac{E}{2}(1 + \nu)
\end{align}

``(on circle, with sign convention)'' holds true because the radius,
\(R\) of Mohr's circle is \(\sigma_{1} = - \sigma_{1}\). Some of these
relationships are not limited to pure shear, because
\(E, G \text{and} \nu\) are material properties.

Expressing three-dimensional stress is important, but can be achieved
with tensor notation. Analysis of stress states can be reduced from
three to two dimensions if assuming plane stress (zero stress in third
axis) or plane strain (zero strain in third axis). This simplified
analysis can be transformed to desired stress states by equations or
Mohr's circle. Recall that Mohr's circle is a graphical representation
of all possible stress states, but it must be drawn accurately that
trigonometric functions might be used effectively.

\hypertarget{elasticity-atomic-bonds}{%
\subsubsection{Elasticity: Atomic Bonds}\label{elasticity-atomic-bonds}}

\hypertarget{introduction-3}{%
\paragraph{\{Introduction\}}\label{introduction-3}}

What preceded was continuum elasticity. What follows is observing the
previously explored behaviors at the atomistic level. These behaviors
determine the way force, stress, and strain occur. Characterizing these
behaviors informs the results of bond strength in materials: we focus on
metals. There exist competing behaviors of attraction and repulsion.

Continuum looks and homogeneous units wherein stress states are
described without knowing the exact material makeup. Here, we can
continue to ignore anisotropic conditions, but must observe the source
of elasticity: atomic bonds. Strength of bonds determined by electronic
characteristics, but can be influenced by external conditions: heat,
magnetic fields, etcetera.

{We assume only planar, uniaxial attractions.}

\hypertarget{atomic-bonding}{%
\paragraph{\{Atomic Bonding\}}\label{atomic-bonding}}

Electronic bonds govern behavior of all material properties. All
mechanical behavior first exhibits elastic behavior. These effects can
be observed with a simple spring model between two atoms.

{Simplified spring model to visually show attractive and repulsive
forces of electronic bonds between atoms.}

\hypertarget{bond-interaction-and-force}{%
\paragraph{\{Bond Interaction and
Force\}}\label{bond-interaction-and-force}}

Two atoms will have an equilibrium separation, \(r_{0}\). We assume this
until some external influence is applied. The minimum bonding energy
also occurs here. Repulsion is caused by the
\textit{Pauli Exclusion Principle} and attraction is \textit{Coulombic}
in nature. \[U_{i} = -\frac{A}{r^{m}} + \frac{B}{R^{n}}\]

{Electronic bonding occurs according to these two curves between any two
atoms.}

\textbf{Example} Potential energy of \(Na^{+}Cl^{-}\), an ionic pair, at
distance \(r\) where \(q_{0} = 1.6e-19 [C]\),
\(\epsilon_{0} = 8.85e-12 [\frac{C}{N-m^{2}}]\), and
\(U_{i} = 1.12 [eV]\). If \(r_{0} = 0.276 [nm]\), then find: \textbf{a]}
the value of B; and \textbf{b]} the total, attractive, and repulsive
forces at \(r = 0.25 [nm]\).

\begin{verbatim}
\begin{align}
    U_{i} &= -\frac{A}{r^{m}} + \frac{B}{r^{n}} \\
    U &= U_{i} - \frac{q^{2}}{4\pi\epsilon_{0}r} + \frac{B}{r^{9}}
\end{align}

\noindent Because $\frac{dU}{dr} = 0$ at $r_{0}$,

\begin{align}
    \frac{dU}{dr} &= 0 = \frac{q^{2}}{4\pi\epsilon_{0}r^{2}} - \frac{9B}{r_{0}^{10}} \\
    \frac{q^{2}}{4\pi\epsilon_{0}r^{2}} &= \frac{9B}{r_{0}^{10}} \\
    \implies B &= \frac{r_{0}^{8}q^{2}}{36\pi\epsilon_{0}} \\
    B &= 8.61e-106 [N-m^{10}]
\end{align}

\noindent $B$, then, is simply plugged into the following equations to find electronic bond forces at not the equilibrium distance, $r = 0.25 [nm]$.

\begin{align}
    F &= \frac{dU}{dr} = F_{A} - F_{R} \\
    F &= \frac{q^{2}}{4\pi\epsilon_{0}r^{2}} - \frac{9B}{r_{0}^{10}} \\
    F_{A} &= \frac{q^{2}}{4\pi\epsilon_{0}r^{2}} = 3.68e-4 [N] \\
    F_{B} &= \frac{9B}{r_{0}^{10}} = -8.13e-9 [N] \\
    \implies F &= 4.44e-9 [N]
\end{align}
\end{verbatim}

\hypertarget{stress-and-strain-in-bonds}{%
\paragraph{\{Stress and Strain in
Bonds\}}\label{stress-and-strain-in-bonds}}

Force is proportional to atomic displacement and change in energy.
Stress requires number of atoms involved in some area by estimating
atomic spacing at equilibrium, \(r_{0}^{2}\). Strain, then, is the
change in spacing divided by the equilibrium spacing. Young's modulus is
a material property of stress over strain. These can be summarized by
the following relations:

\begin{align}
    F &= \frac{dU_{i}}{dr} \\
    d\sigma &= NdF \\
    d\sigma &= \frac{dF}{r_{0}^{2}} \\
    d\epsilon &= \frac{dr}{r_{0}} \\
    E &= \frac{Am(n-m)}{r_{0}^{m + 3}} \\
    A &= \frac{q^{2}}{4\pi\epsilon_{0}} \\
    E &= \frac{kq^{2}}{r_{0}^{4}}
\end{align}

{Stronger bonding can effect melting point, stiffness, etcetera.
Understanding these effects informs how higher length scales behave
under some condition.}

\hypertarget{summary-2}{%
\paragraph{\{Summary\}}\label{summary-2}}

Bulk properties come from atomic bonds. Treatment of elasticity at
atomic scale determines balances of attractive and repulsive forces;
therefore, external forces cause a shift from the equilibrium state.
Bond properties can be adjusted to preference with some processing
technique: e.g.~alloying.

\hypertarget{exam-review}{%
\subsection{*\{Exam Review\}}\label{exam-review}}

\textbf{Miller Indices} When drawing Miller indices--coordinates,
vectors, and planes--keep to Fig. \ref{#fig:milled_index_axis}
convention. This makes grading quicker, because a different orientation
is not necessarily wrong but are harder to grade.

{Keep cubes with z-vertical and hcp with c-vertical.}

Atomic packing factors will be used in the exam (per the provided table
from earlier lectures).
\textbf{The content of the homework is the template for the exam: no new content on exams.}
WRT to calculating unit cell mass, follow this algorithm example for
iron (Fe):

\begin{align}
\rho &= \frac{nA}{V_{c}N_{A}} \\
\rho_{Fe} &= (55 amu) (\frac{g}{mol}/1 amu) / (N_{A}) \\
\rho_{Fe} &= 3.053e-22 g/atom \\
\rho_{Fe, bcc} &= \rho_{Fe}*2/a_{0}^{3}
\end{align}

\textbf{Tensorial vs.~Engineering Strain}

{Engineering strain keeps deformation in one axis. Actual strain is half
that, but in two axes.}

Because we know the relation, \(\epsilon_{ij} = \frac{\gamma_{ij}}{2}\),
we know \(G = \frac{\tau}{\gamma} = \frac{E}{2(1 + \nu)}\). We can then
know the deformation on some axis:
e.g.~\(\gamma_{12} = \frac{\sigma_{12}}{G} = -37.1e-6\). But tensor
notation for shear strain, \(\gamma_{shear} = \frac{\gamma_{calc}}{2}\).

\textbf{Calculating Strains from Poisson's Ratio} Because the
compression test is performed in a single axis to squish some specimen,
the other lateral strains can be calculated by assuming plane stress:
zero stress in the direction normal to the thinnest dimension.

{}

Pure shear occurs when \(\sigma_{22} = -\sigma_{11}\). The shear stress,
\(\tau\) is the radius of Mohr's circle. The principal stress are the
maximum and minimum of Mohr's circle. A stress tensor is the sum of the
hydrostatic and deviatoric stress tensors:
\([\sigma] = [\sigma_{hydro}] + [\sigma_{dev}]\) Hydrostatic
stress--\(\sigma_{hydro} = \frac{\sigma_{11} + \sigma_{22} + \sigma_{33}}{3}\)--goes
on the diagonal of the tensor and will cause a change volume, but not
the object's shape. Deviatoric stress--elements not on the
diagonal--will change an object's shape, but not its volume.

{Remember that maximum shear on Mohr's circle--at
\(90\text{\textdegree}\)--is twice the angle in real space: i.e.~pure
shear occurs \(45\text{\textdegree}\) of the actual part.}

\hypertarget{plasticity}{%
\subsection{Plasticity}\label{plasticity}}

\hypertarget{introduction-4}{%
\subsubsection{Introduction}\label{introduction-4}}

Plasticity is sometimes involved with engineering design. It may also
prove integral to performance. This chapter explores the importance the
stress-strain behavior and the effects thereof.

Materials always follow elastic to plastic deformation before ultimate
failure. Most applications will operate within the elastic region
(ceramics with narrow elastic regimes), but accommodations for
work-hardening may be considered. Not all materials work-harden the same
way. Two yield points in materials: \textit{first yield} is the elastic
limit, and \textit{ultimate strength} is ultimate plasticity.
\textbf{Plasticity} is imperative for processing and performance of
materials. Think of Dr.~Atwater's lawn mower!

\textbf{Mechanical Testing} Determines mechanical properties for
materials: such as various tension or compression. Tension is the most
popular, but all give same information: \textit{stress-strain} curve.

\hypertarget{engineering-and-true-stress-and-strain}{%
\paragraph{\{Engineering and true stress and
strain\}}\label{engineering-and-true-stress-and-strain}}

Recall that engineering stress is from \(A_{0}\) and true uses
\(A_{i}\). \textit{Plastic deformation is volume conservative}, which
allows calculating true stress and strain from the engineering values.
Stress-strain curves relates initial conditions to overall performance.

{True stress-strain gives more accurate understanding of stress states,
but can be more difficult to interpret.}

\textbf{Q\&A}
\textbf{DK: Is conversion from engineering to true stress-strain meaningful with necking?}
\textit{It depends. We will discuss this later.}

\hypertarget{work-hardening-basics}{%
\paragraph{\{Work-Hardening Basics\}}\label{work-hardening-basics}}

\textit{Ludwik-Hollomann} equations:
\(\sigma = \sigma_{0} + K\epsilon^{n}\). \(\sigma_{0}\) is the yield
stress, \(K\) is experimentally found
(\(\nicefrac{G}{100}-\nicefrac{G}{1000}\)); \(\epsilon\) is true strain;
and, \(n\) is some work hardening coefficient (0.2-0.5).

\textbf{Example} Use the \textit{Ludwik-Hollomann} equation to determine
work hardening exponent, \(n\) in an lloy of true strain at 0.1 and true
stress = 415 MPa. Assume \(K = 1035 MPa\) and \(\sigma_{0} = 0\).

\begin{align*}
    \sigma &= \sigma_{0} + K\epsilon^{n} \\
    log[\sigma &= \sigma_{0} + K\epsilon^{n}] \\
    log(\sigma) &= log(K) + n*log(\epsilon) \\
    \frac{log(\sigma) - log(K)}{log(\epsilon)} &= n \\
    \implies n &= \frac{log(415) - log(1035)}{log(0.1)} \\
    n &\approx 0.397
\end{align*}

\textit{The greater, $n$, the more work-hardening can occur.}

\textbf{Refined Methods} \textit{L=H} has limits; therefore,
\textit{Voce} equations adds asymmetry. \textit{Johnson-Cook} expands
with strain-rate and temperature dependence:
\(\sigma = (\sigma_{0} + K\epsilon^{n})\bigl(1 + C*ln(\frac{\dot{\epsilon}}{\dot{\epsilon_{0}}})\bigr)\bigl[1 - \bigl(\frac{T - T_{r}}{T_{m} - T_{r}}\bigr)^{m}\bigr]\).
Each term in the \textit{Johnson-Cook} equation represents a different
failure mechanism.

\textbf{Volume Conservation} Volume is assumed to be constant through
deformation; however, volume is not always assumed constant in the
elastic region. \textit{Poisson's Ratio}, \(\nu\) is constant in the
elastic region, but varies in the plastic region. True and engineering
stress-strains considered equivalent up to elastic limit.

{Steels have upper and lower yield limits to break dislocations apart.}

\hypertarget{summary-3}{%
\paragraph{\{Summary\}}\label{summary-3}}

Plasticity introduces new requirements to calculate stress and strain.
Elastic portion considered inconsequential and plastic deformation is
volume constant.

\hypertarget{tensile-curve-parameters-necking-and-strain-rate}{%
\subsubsection{Tensile Curve Parameters, Necking, and Strain
Rate}\label{tensile-curve-parameters-necking-and-strain-rate}}

Features of the stress-strain curve indicate when necking occurs, and
give insight when correction factors apply: only when cylindrical
samples neck. New equations for material behavior after necking.
Strain-rate affects material response as well.

\hypertarget{introduction-5}{%
\paragraph{\{Introduction\}}\label{introduction-5}}

Tensile testing is the most basic form to determine material properties.
Parameters of test affect outcome: temperature, etcetera.

{\textit{A:} \(0.2\%\) strain offset yield stress. \textit{B:} upper
yield. \textit{C:} lower yield. \textit{D:} proportional limit.
\textit{D`:} ultimate tensile strength (UTS). \textit{E:} rupture
stress. \textit{F:} non-uniform plastic strain limit. \textit{G:}
rupture strain (strain to failure). \textit{H:} lower yield region. The
area under the elastic region is the \textit{modulus of resilience} and
\textit{toughness} is area under entire curve.}

The more dislocations/impurities, the more local stresses exist, and
more global force is required to overcome the sum of the internal
stress: this is true for plain carbon steels.

\textbf{Items of Note} Yielding is preceded by \textit{micro-yielding},
where dislocation motion can occur below traditional yield stress.
Upper-lower yield behavior is largely seen in plain, low-carbon steels.
Vacancy and dislocation pinning resist initial yielding. Strain rate
will modify the tensile curve and can obscure the upper-lower yield
phenomenon when present.

\hypertarget{necking}{%
\paragraph{\{Necking\}}\label{necking}}

Occurs when localized deformation begins to dominate the strain:
\textit{void nucleation, coalescence, and growth}. This is also known as
\textit{plastic instability} and is defined by \textbf{Considere}
criterion: increase in stress relative to strain (work-hardening)
reaching a maximum in the engineering stress-strain curve.
\textit{The higher the strain exponent, the more strain you get out of the material.}
Using this criterion, substituting true stress-strain into the
derivative and apply the Hollomann equation, you get the relationship
\(\epsilon_{u} = n\), where \(\epsilon_{n}\) is maximum, uniform plastic
strain.

Work-hardening exponent from \textit{engineering} stress-strain curve.
Work-hardening decreases during increase of plastic strain until that
point at which necking occurs. A metal unable to work-harden immediately
reaches the point of necking after yielding, which is consistent with
equations. \textit{Work-softening}--the more it deforms, the easier it
can be deformed (not necessarily from reduced area)--is possible under
extreme conditions.
\textbf{Think of void nucleation, coalescence, and growth!}

\hypertarget{stress-strain-and-necking}{%
\paragraph{\{Stress-strain and
Necking\}}\label{stress-strain-and-necking}}

After necking, instantaneous cross-sectional area must be continuously
determined. Neck acts as a ``second'', miniature tensile specimen, so
it's strain-rate is higher from shorter length. Irregular geometry of
neck also introduces triaxial flow stress. Magnitude of transverse
stresses depends on sample and neck geometry and strain-rate.

\hypertarget{bridgman-correction}{%
\subparagraph{\{Bridgman Correction\}}\label{bridgman-correction}}

\begin{verbatim}
This only applies to \textit{cylindrical} samples, because sample area and neck radius must be known.

$$\sigma = \frac{\sigma_{avg}}{(1 + 2\frac{R}{r_{n}})ln(1 + \nicefrac{r_{n}}{2R})}$$

\noindent $R$ is radius of curvature of the neck, and $r_{n}$ is the cross-sectional radius at thinnest part of neck.
The further away from necking (higher strains), this correction factor increases.
\end{verbatim}

\hypertarget{state-of-stress-in-deformation}{%
\subparagraph{\{State of Stress in
Deformation\}}\label{state-of-stress-in-deformation}}

\begin{verbatim}
Necking is onset of failure in a non-uniform fashion.
Applies only to tensile testing, because compression samples \textit{barrel}.
Necking can be suppressed to achieve higher strains in more complex stress states.
\end{verbatim}

{Wire drawing shows ability to achieve 7.4 true strain of copper.}

\hypertarget{strain-rate}{%
\paragraph{\{Strain Rate\}}\label{strain-rate}}

Increasing strain-rate work-hardens faster:
\textbf{void nucleation controls}!! This parameter is part of
stress-strain rate relationship: \(\sigma = K\dot{\epsilon}^{m}\). \(m\)
can be found from a jump test between two strain-rates using the
Hollomann equation:
\(m = \frac{ln(\nicefrac{\sigma_{2}}{\sigma_{1}})}{ln(\nicefrac{\dot{\epsilon_{2}}}{\dot{\epsilon_{1}}})}\).

\textbf{Strain-rate In Practice} Strain-rate can vary from
\(10^{-6}s^{-1}\) to \(10^{6}s^{-1}\). Tensile tests usually within
\(10^{-4}s^{-1}\) to \(10^{-1}s^{-1}\). High strain-rate (Hoppy bar) and
creep or stress relaxation tests invaluable to determine material
behavior under extreme conditions. Creep and stress relaxation tests
incorporate higher temperatures, which is important to material
performance.

Strain-rate affects material behavior. Increasing strain-rate often
increases yield stress and the work-hardening rate. Typically,
\(0.02<m<0.2\) for \(0-0.9T_{H}\).

\textbf{Super-plastic Behavior} Effective strain-rate in necking area
increases. Positive strain-rate sensitivity implies an increased stress
and the yield point will shift to the weaker section. Same concept that
assists in distributing strain across tensile specimen length in
work-hardening.

\hypertarget{summary-4}{%
\paragraph{\{Summary\}}\label{summary-4}}

Stress-strain curve contains much information, but not all can be known
from only the engineering curve. Necking changes specimen geometry,
which affects test results. Stress states may control onset and
progression of necking, which may allow for much higher strains before
failure. Strain-rate affects material response and is dependent on
processing and performance applications.

\hypertarget{compression-and-hardness}{%
\subsubsection{Compression and
Hardness}\label{compression-and-hardness}}

\hypertarget{introduction-6}{%
\paragraph{\{Introduction\}}\label{introduction-6}}

Many applications use compressive stresses; therefore, testing for
compression is more relevant. Some materials perform well under
compression and not tension. Simpler and more cost-effective to perform
compression tests; therefore, metal alloy research heavily utilizes this
method.

\hypertarget{practical-considerations}{%
\paragraph{\{Practical
Considerations\}}\label{practical-considerations}}

Compression is simple but requires care to ensure good data. Plate
alignment and sample parallelism and flatness are imperative. Lubricant
between plates and sample reduce barreling.
\textit{Plates should be much harder than the sample.} Machine
compliance must be removed from measured stress-strain curve, especially
at lower strains. Can also measure strain from video recordings and
point markers on sample.

\hypertarget{compression-curve}{%
\paragraph{\{Compression Curve\}}\label{compression-curve}}

Stress-strain opposite from tensile curve. Barreling is source of
non-uniform plastic strain.

{True stress-strain moves the curve down and to the right from the
engineering stress-strain curve, which is opposite that seen for tension
testing.}

\noindent Effects of barreling most pronounced at strains exceeding 0.4;
therefore, compression testing usually limited to less than that.
Friction is very important when initial \(\frac{height}{diameter}\) is
reduced: if too tall and thin, the specimen will buckle first.

\hypertarget{compression-failure}{%
\paragraph{\{Compression Failure\}}\label{compression-failure}}

If perfectly striked, stress state greatly varies through specimen.
Although extreme, ductility allows for stress-strain behavior off the
central axis. Non-uniform stress occurs, then, across top and bottom
surfaces (friction hill).

\[ p = \sigma_{0}\exp{nicefrac{2\mu(a - r)}{h}} \]
\{\#eq:friction\_hill\_pressure\}

\noindent \(\mu\) is coefficient of friction, \(r\) is distance from
center, \(a\) is radius of sample, and \(h\) is height of sample.

{Friction hill seen at top and bottom surfaces of sample when
barreling.}

\textbf{Example} If \(\frac{height}{diameter} = 2\), then what is
\(p_{max}\)?

\begin{align*}
    \frac{l}{d}     &= 2 \\
    p   &= \sigma_{0}\exp{\nicefrac{2\mu(a - r)}{h}} \\
    &= \sigma_{0}\exp{2(0.15)(\frac{a}{h})}, \frac{d}{h} = \frac{1}{2}(\frac{\nicefrac{a}{2}}{h}) = \frac{1}{2} \longrightarrow \frac{a}{h} = \frac{1}{4}
    \implies p  &= \sigma_{0}\exp{2(0.15)(\frac{1}{4})} \\
    &= \sigma_{0}\exp{\frac{0.3}{4}}
\end{align*}

\hypertarget{bauschinger-effect}{%
\paragraph{\{Bauschinger Effect\}}\label{bauschinger-effect}}

If you pull something in tension, then switch to compression, the yield
point will lower from tension to compression. The strain in the material
from tension weakens the material and causes a lower yield point: the
\textit{Bauschinger effect}.

\hypertarget{hardness-testing}{%
\paragraph{\{Hardness Testing\}}\label{hardness-testing}}

\textbf{Hardness is a material's resistance to plastic flow by indentation.}
Scale of indentation varies with load, which is standardized with the
indenter while its displacement is measured.

\textbf{Brinell} Uses a hard sphere of known dimension and known force
to measure the size of indent. Amount of applied force changes size of
indention.

\[ HB = \frac{P}{\pi Dh} = \frac{2P}{\pi D(D - \sqrt{D^{2} - d^{2}})}\]
\{\#eq:brinell\_hardness\_number\}

\[ HB_{Meyer} = \frac{4P}{\pi d^{2}} \] \{\#eq:meyer\_hardness\_number\}

\textbf{Rockwell}

\end{document}
